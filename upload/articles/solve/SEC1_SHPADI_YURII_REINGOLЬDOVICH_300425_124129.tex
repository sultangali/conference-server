\documentclass[12pt]{article}

\usepackage{libertine}
\usepackage[T1]{fontenc}
\usepackage[utf8]{inputenc}
\usepackage{microtype}
\usepackage[english, russian]{babel}
\usepackage{amsmath}
\usepackage{graphicx}
\usepackage{geometry}
\usepackage{fancyhdr}
\usepackage{natbib}
\usepackage{xcolor}
\usepackage{titlesec}
\usepackage{booktabs}
\usepackage[colorlinks=true,linkcolor=black,citecolor=black,urlcolor=black]{hyperref}

\geometry{
    margin=1in,
    headheight=12pt
}
\setcitestyle{square}

\pagestyle{fancy}
\fancyhf{}
\renewcommand{\headrulewidth}{0.4pt}
\renewcommand{\footrulewidth}{0.4pt}
\fancyhead[L]{\small Актуальные вопросы междисциплинарных научных исследований}
\fancyhead[R]{\small\thepage}
\fancyfoot[C]{\small Караганда, 2025}

\titleformat{\section}
    {\normalfont\large\bfseries}{\thesection}{1em}{}
\titlespacing*{\section}
    {0pt}{2.5ex plus 1ex minus .2ex}{1.5ex plus .2ex}

\makeatletter
\renewcommand{\maketitle}{%
    \begin{center}
        \Large\@title

        \vspace{0.4cm}
        \large\@author

        \vspace{0.5cm}
        \normalsize\textit{\textsuperscript{1} Институт математики и математического моделирования, Алматы, Казахстан\\
          E-mail: yu-shpadi@yandex.kz\\}
%          \textsuperscript{2}Название университета, город, страна\\
%          E-mail: test@example.com\\
%          \textsuperscript{3}Название университета, город, страна\\
%          E-mail: test@example.com\\}
    \end{center}
}
\makeatother

% Paper info
\title{ТЕМПЕРАТУРА КОНТАКТА ВЫКЛЮЧАТЕЛЯ ПОД ДЕЙСТВИЕМ ТЕПЛОВОГО ПОТОКА ЭЛЕКТРИЧЕСКОЙ ДУГИ, ПОДЧИНЯЮЩЕГОСЯ ЗАКОНУ НЬЮТОНА}

% Uncomment and update the following line upon acceptance
% \author{Author Name(s)\\ % Institution Name(s)}

%Comment the following line upon acceptance
\author{Шпади Ю.Р.\textsuperscript{1}} %, Второй автор\textsuperscript{2}, Третий автор\textsuperscript{3}}


\date{}

\begin{document}

\maketitle

Рассматривается задача прогрева электрода коммутационного аппарата под воздействием электрической дуги и внутреннего источника тепла. Математическая модель базируется на квазистационарном уравнении теплопроводности со сферической симметрией 
\begin{equation}\label{Shpadi-eq1}
\frac{1}{r^2}\frac{\partial}{\partial {r}}\left(r^2 \lambda(r,t,\theta)\frac{\partial {\theta}}{\partial{r}}\right) +\frac{q(r,t)}{r^4} = 0,\quad \theta = \theta(r,t),     
\end{equation}
заданном в области со свободной границей $\Omega = \left\{ (r,t) | 0<b<r<\alpha(t) < \infty, \ 0<t<t_a \right\}$, при краевых условиях:
\begin{equation}\label{Shpadi-eq2}
  \alpha(0) = b,
\end{equation}
\begin{equation}\label{Shpadi-eq3}
  -\left. \lambda(r,t;\theta) \frac{\partial \theta}{\partial r} \right|_{r=b}  = h\left(\theta_{arc}(t)-\theta(b,t)\right),
  \quad \theta_{arc}(t) > \theta((b,t), \ t\in[0,t_a],
\end{equation}
\begin{equation}\label{Shpadi-eq4}
  \theta(\alpha(t),t) = \theta_{\alpha} (t),
\end{equation}
\begin{equation}\label{Shpadi-eq5}
  -\left. \lambda(r,t;\theta) \frac{\partial \theta(r,t)}{\partial {r}} \right|_{r=\alpha(t)} = L(t;\theta_{\alpha} (t)) \frac{d \alpha(t)}{d t}.
\end{equation}

В начальный момент $t=0$ область $\Omega$ ввиду условия \eqref{Shpadi-eq2} вырождена (обращается в точку).  Для расчета закона движения свободной границы $\alpha(t)$  и температурного поля $\theta(r,t)$  внутри области  $\Omega$  задача \eqref{Shpadi-eq1} –- \eqref{Shpadi-eq5} редуцируется к эквивалентной системе двух нелинейных интегральных уравнений с переменными пределами интегрирования.     

Доказано, что при условии непрерывности и положительности функций $\lambda(r,t;\theta)$ , $P(t)$, $q(r,t)$ , $L(t;\theta_{\alpha}(t)$ а также непрерывности функции $\theta_{\alpha}(t)$  и липшицевости функции $\lambda(r,t;\theta)$  по переменной $\theta$, система интегральных уравнений однозначно разрешима и ее решение является решением краевой задачи \eqref{Shpadi-eq1} –- \eqref{Shpadi-eq5}. 

Разработан численный алгоритм итеративного решения системы интегральных уравнений. Проведен вычислительный эксперимент на тестовой краевой задаче с точным аналитическим решением и представлены результаты сходимости итераций в метрике непрерывных функций полных пространств.  

\textbf{Благодарности:}
 Данное исследование финансируется Комитетом науки Министерства науки и высшего образования Республики Казахстан (грант № АР19675480).

\textbf{Ключевые слова:} нелинейное уравнение теплопроводности, сферическая симметрия, краевые условия, проблема Стефана, интегральные уравнения, электрические контакты.

% BIBLIOGRAPHY (can be omitted)
% Please use the following commands to format the reference list
\end{document}


