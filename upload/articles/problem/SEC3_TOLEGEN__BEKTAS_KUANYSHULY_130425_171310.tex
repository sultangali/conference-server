\documentclass[12pt]{article}

\usepackage{libertine}
\usepackage[T1]{fontenc}
\usepackage[utf8]{inputenc}
\usepackage{microtype}
\usepackage[english, russian]{babel}
\usepackage{amsmath}
\usepackage{graphicx}
\usepackage{geometry}
\usepackage{fancyhdr}
\usepackage{natbib}
\usepackage{xcolor}
\usepackage{titlesec}
\usepackage{booktabs}
\usepackage[colorlinks=true,linkcolor=black,citecolor=black,urlcolor=black]{hyperref}

\geometry{
    margin=1in,
    headheight=12pt
}
\setcitestyle{square}

\pagestyle{fancy}
\fancyhf{}
\renewcommand{\headrulewidth}{0.4pt}
\renewcommand{\footrulewidth}{0.4pt}
\fancyhead[L]{\small Current issues in mathematics}
\fancyhead[R]{\small\thepage}
\fancyfoot[C]{\small Караганда, 2025}

\titleformat{\section}
    {\normalfont\large\bfseries}{\thesection}{1em}{}
\titlespacing*{\section}
    {0pt}{2.5ex plus 1ex minus .2ex}{1.5ex plus .2ex}

\makeatletter
\renewcommand{\maketitle}{%
    \begin{center}
        \Large\@title

        \vspace{0.4cm}
        \large\@author

        \vspace{0.5cm}
        \normalsize\textit{\textsuperscript{1} Karaganda Buketov University, Universitetskaya ul. 28, Karaganda, Kazakhstan, 100026\\
          E-mail: bektas.tolegen99@gmail.com\\
          \textsuperscript{2}Karaganda Buketov University, Universitetskaya ul. 28, Karaganda, Kazakhstan, 100026\\
          E-mail: nariman.tokmagambetov@gmail.com\\}
    \end{center}
}
\makeatother

% Paper info
\title{Existence of solutions in boundary value problems for nonlinear equations involving q-analogs of fractional derivatives}

% Uncomment and update the following line upon acceptance
% \author{Author Name(s)\\ % Institution Name(s)}

%Comment the following line upon acceptance
\author{Tolegen B.K.\textsuperscript{1}, Tokmagambetov N.S.\textsuperscript{2}}


\date{}

\begin{document}

\maketitle

This study begins by presenting the key definitions and preliminary concepts, including the fundamental principles of $q$-calculus, which form the theoretical foundation of the work. Detailed discussions can be found in \cite{CK2000}, \cite{E2002} and \cite{A1969}. This work operates under the standing assumption that
$0<q<1$.

Let $\alpha \in \mathbb{R}$. Then a $q$-real number $[\alpha ]_q$ is defined by
$$
[\alpha ]_{q}=\frac{1-q^{\alpha }}{1-q},
$$
where $\mathop{\lim }\limits_{q\rightarrow 1}\frac{1-q^{\alpha }}{1-q}%
=\alpha $.


We introduce for $k\in\mathbb{N}$:
\begin{eqnarray*}
(a;q)_0=1,  \; (a;q)_n=\prod\limits_{k=0}^{n-1}\left(1-q^ka\right),  \; (a; q)_\infty = \lim\limits_{n\rightarrow\infty}(a,q)_n, \;
(a;q)_\alpha=\frac{(a;q)_\infty}{(q^\alpha a;q)_\infty}.
\end{eqnarray*}



For any two real numbers $\alpha$ and $\beta$, we have
\begin{eqnarray*}
\left(a-b\right)^\alpha_q\left(a-q^\alpha{b}\right)^\beta_q=\left(a-b\right)^{\alpha+\beta}_q.
\end{eqnarray*}

The gamma function $\Gamma_q(x)$ is defined by
\begin{eqnarray*}
\Gamma_q(x)=\frac{(q; q)_\infty }{(q^x; q)_\infty }(1-q)^{1-x}, \;\;\;
\end{eqnarray*}
for any $x > 0$. Moreover, it yields that
\begin{eqnarray*}
\Gamma_q(x)[x]_q=\Gamma_q(x+1).
\end{eqnarray*}

The $q$-analogue differential operator $D_{q}f(x)$ is
\begin{eqnarray*}
D_{q}f(x)=\frac{f(x)-f(qx)}{x(1-q)},
\end{eqnarray*}
and the $q$-derivatives $D^{n}_q(f(x))$ of higher order are defined inductively as follows:
\begin{eqnarray*}
D_q^0(f(x))=f(x), \;\;\; D^{n}_q(f(x))=D_q\left(D_q^{n-1}f(x)\right), (n=1,2,3, \dots).
\end{eqnarray*}

The $q$-integral of a function $f$ defined in the interval $[0,b]$ is defined by the expression
\begin{eqnarray*}
\left(I_q f\right)(x)=\int\limits_0^x f(x) \mathrm{d}_q x=x(1-q) \sum_{n=0}^{\infty} f\left(x q^n\right) q^n, \quad x \in[0, b].
\end{eqnarray*}

The \(q\)-Beta function is defined for any \(\alpha, \beta > 0\) as follows:
\begin{eqnarray*}
B_{q}(\alpha, \beta)=\frac{\Gamma_{q}(\alpha)\Gamma_q{(\beta)}}{\Gamma_q(\alpha+\beta)}=\int\limits_{0}^{1}x^{\alpha-1}(1-qx)^{\beta-1}_qd_qx.
\end{eqnarray*}

\textbf{Definition 1.} \cite{AM2012}.
Let \(\alpha \geq 0\) and \(f\) be a function defined on \([0,1]\). The Riemann-Liouville fractional \(q\)-integral is expressed as \(\left( I_{q}^{\alpha }f \right)\left( x \right)\), and are defined by
\begin{eqnarray*}
\left(I^\alpha_{q,0+}f\right)(x)=\frac{1}{\Gamma_q(\alpha)}\int\limits_0^x(x-qs)_q^{(\alpha-1)}f(s)d_qs.
\end{eqnarray*}


\textbf{Definition 2.}
The Riemann-Liouville fractional \(q\)-derivative of order \(\alpha \geq 0\) is defined as \(\left( D_{q}^{\alpha }f \right)\left( x \right)\), and
\begin{eqnarray*}
\left(D_q^\alpha f\right)(x)=\left(D_q^{[\alpha]} I_q^{[\alpha]-\alpha} f\right)(x), \quad \alpha>0,
\end{eqnarray*}
where $[\alpha]$ is the smallest integer greater than or equal to $\alpha$.

\textbf{Lemma 1.} \cite{YU2013}.
Let \( g(x) \in L[0,1] \) and \( 1 < \alpha \leq 2 \), then there exists a unique solution for
\begin{eqnarray}\label{additive2.9}
D_{q, 0+}^\alpha y(x)+f(x,y(x))=0, \quad 0<x<1,  \;\;0<q<1,
\end{eqnarray}
\begin{eqnarray}\label{additive2.10}
y(0)=0, \quad D_{q,0+}^\beta y(1)=a D_{q,0+}^\beta y(\tau)+\lambda
\end{eqnarray}
is
\begin{eqnarray*}
y(x)=\int\limits_0^1 G_q(x, s) f(x,y(x)) \mathrm{d}_q s+\lambda x^{\alpha-1}\frac{\Gamma_q(\alpha+\beta)}{\Gamma_q(\alpha)},
\end{eqnarray*}
where
\[
G_q(x, s) =
\begin{cases}
    \frac {d x^{\alpha - 1} (1 - qs)^{(\alpha - \beta - 1)}_q - a d x^{\alpha - 1}(\tau -qs)^{(\alpha - \beta - 1)}_q - (x - qs)^{(\alpha - 1)}_q}{\Gamma_q(\alpha)}, & 0 \leq s \leq \min\{x, \tau\} < 1, \\[5pt]
    \frac{d x^{\alpha - 1} (1 - qs)^{(\alpha - \beta - 1)}_q -(x - qs)^{(\alpha - 1)}_q}{\Gamma_q(\alpha)}, & 0 \leq \tau \leq s \leq x \leq 1, \\[5pt]
    \frac{d x^{\alpha - 1} (1 - qs)^{(\alpha - \beta - 1)}_q - a d x^{\alpha - 1} (\tau - qs)^{(\alpha - \beta - 1)}_q}{\Gamma_q(\alpha)}, & 0 \leq x \leq s \leq \tau \leq 1, \\[5pt]
    \frac{d x^{\alpha - 1} (1 - qs)^{(\alpha - \beta - 1)}_q}{\Gamma_q(\alpha)}, & \max\{x, \tau\} \leq s \leq 1,
\end{cases}
\]
for which
$
d = \left(1 - a \tau^{\alpha - \beta - 1}\right)^{-1},
$
$0 \leq \beta \leq 1$, $0 \leq a \leq 1$, $\lambda\geq\lambda_0>0$, $\tau \in (0,1)$, $a \tau^{\alpha-\beta-2} \leq 1-\beta$, $0 \leq \alpha-\beta-1$  and $f:[0,1] \times[0, \infty) \rightarrow[0, \infty).$

In this work, we assume that the function
\[
f: [0,1] \times [0,\infty) \to [0,\infty)
\]
the following Caratheodory conditions are satisfied

\((H1)\):
\begin{enumerate}
    \item[(i)] For every fixed \(t \in [0,\infty)\), the mapping \(x \mapsto f(x,t)\) is Lebesgue measurable on \([0,1]\);
    \item[(ii)] For every fixed \(x \in [0,1]\), the mapping \(t \mapsto f(x,t)\) is continuous on \([0,\infty)\).
\end{enumerate}


\textbf{Theorem 1.}
Assume that (H1) is satisfied and that there exists a real-valued function \(h(x) \in L[0,1]\) such that
\begin{eqnarray*}
|f(x,y) - f(x,z)| \leq h(x) |y - z|,
\end{eqnarray*}
for almost every   $x\in [0,1]$  and all $y, z \in [0,\infty)$.
If
\begin{eqnarray*}
0 < \int\limits_0^1 G_q(s,s)\, h(s) \, d_qs < 1,
\end{eqnarray*}
then, the boundary value problem (BVP) defined by (1) and (2) on \([0,1]\) admits a unique positive solution.

\textbf{Lemma 2.}
Assume that condition (H1) holds and that there exist two nonnegative, real-valued functions \(m\) and \(n\) in \(L[0,1]\) such that
\begin{eqnarray*}
f(x, t) \leq n(x)+m(x) t, \quad
\end{eqnarray*}
for almost all $x \in[0,1]$ and all $t \in[0, \infty)$.
Hence, the operator \(T: P \rightarrow P\), as defined in (Lemma 1), is completely continuous.


\textbf{Theorem 2.}
Assume that all the conditions of lemma 2 are satisfied. If
$$
\int\limits_0^1 G_q(s, s) m(s) \, \mathrm{d}_q s < 1,
$$
then the boundary value problem (1) and (2) admits at least one solution.


\textbf{Acknowledgments:}
This research is funded by the Science Committee of the Ministry of Science and Higher Education of the Republic of Kazakhstan (Grant no.
AP22687134, 2024-2026)


\begin{thebibliography}{00}\label{ref:ref}
\bibitem {CK2000}  P.~Cheung and V.~Kac, \emph{Quantum calculus} (Inc., Ann Arbor, MI, USA, 2000).
\bibitem {E2002} T.~Ernst, Doctoral thesis, A new method of $q$-calculus, Uppsala university, Uppsala 2002.
\bibitem {A1969} W.~Al-Salam, \textquotedblleft Some fractional $q$-integrals and $q$-derivatives,\textquotedblright Proc. Edinb. Math. Soc. \textbf{15}, 135--140 (1966/1967).
\bibitem {AM2012}  M.H.~Annaby and Z.S.~Mansour, \emph{$q$-fractional calculus and equations} (Springer, Heidelberg, 2012).
\bibitem {YU2013}   C.~Yu and J.~Wang,  \textquotedblleft Positive solutions of nonlocal boundary value problem for high-order nonlinear fractional q-difference equations,\textquotedblright B Abstr. Appl. Anal., Article ID928147, (2013).
\end{thebibliography}
\end{document}



