\documentclass[12pt]{article}

\usepackage{libertine}
\usepackage[T1]{fontenc}
\usepackage[utf8]{inputenc}
\usepackage{microtype}
\usepackage[english, russian]{babel}
\usepackage{amsmath}
\usepackage{graphicx}
\usepackage{geometry}
\usepackage{fancyhdr}
\usepackage{natbib}
\usepackage{xcolor}
\usepackage{titlesec}
\usepackage{booktabs}
\usepackage[colorlinks=true,linkcolor=black,citecolor=black,urlcolor=black]{hyperref}
\newtheorem{theorem}{\indent Theorem}
\newtheorem{lemma}{\indent Lemma}
\newtheorem{proposition}{\indent Proposition}
\newtheorem{corollary}{\indent Corollary}
\newtheorem{definition}{\indent Definition}

\geometry{
    margin=1in,
    headheight=12pt
}
\setcitestyle{square}

\pagestyle{fancy}
\fancyhf{}
\renewcommand{\headrulewidth}{0.4pt}
\renewcommand{\footrulewidth}{0.4pt}
\fancyhead[L]{\small Актуальные вопросы междисциплинарных научных исследований}
\fancyhead[R]{\small\thepage}
\fancyfoot[C]{\small Караганда, 2025}

\titleformat{\section}
    {\normalfont\large\bfseries}{\thesection}{1em}{}
\titlespacing*{\section}
    {0pt}{2.5ex plus 1ex minus .2ex}{1.5ex plus .2ex}

\makeatletter
\renewcommand{\maketitle}{%
    \begin{center}
        \Large\@title

        \vspace{0.4cm}
        \large\@author

        \vspace{0.5cm}
        \normalsize\textit{\textsuperscript{1}L.N. Gumilyov Eurasian National University, Astana, Kazakhstan\\
          E-mail: azhar.abekova@gmail.com\\
          \textsuperscript{2}Institute of Mathematics of the
 Czech Academy of Sciences, Praha, Czech Republic\\
          E-mail: gogatish@math.cas.cz\\
          \textsuperscript{3}L.N. Gumilyov Eurasian National University, Astana, Kazakhstan\\
          E-mail: bokayev2011@yandex.kz\\
          \textsuperscript{4}Kirikkale University, Kirikkale, Turkey\\
          E-mail: tugceunver@kku.edu.tr\\}
          
    \end{center}}

\makeatother

% Paper info
\title{WEIGHTED INEQUALITIES FOR SUPERPOSITION OF OPERATORS}

% Uncomment and update the following line upon acceptance
% \author{Author Name(s)\\ % Institution Name(s)}

%Comment the following line upon acceptance
\author{Abek A.N.\textsuperscript{1}, Gogatishvili A.\textsuperscript{2}, Bokayev N.A\textsuperscript{3}, Unver T.\textsuperscript{4}}

\date{}

\begin{document}

\maketitle

In this paper we consider the superposition of three operators: Copson, Hardy and Tandori. Denote by $\mathfrak{M}^+(0,\infty)$ the set of all non-negative measurable functions on $(0,\infty)$.
%and $\mathfrak{M}^{\uparrow}(0,\infty)$ is the class of non-decreasing elements of $\mathfrak{M}^+(0,\infty)$.

Let $1\leq p<\infty$, $0<q<\infty$, $u$, $v$ and $w$ are weights, (i.e. locally integrable non-negative functions on $(0,\infty)$),  $\varphi$ is strictly increasing function $(0,\infty)$, and $\frac{\varphi}{U}$ is decreasing on $(0,\infty)$,  where
\[U(s)=\int_{0}^{s}u(t)dt\]

Our goal in this paper is to characterize  the following inequality
\begin{equation}\Bigg(\int_{0}^{\infty}\left({\sup_{t<s<\infty}}\frac{1}{\varphi(s)}\int_{0}^{s} \left(\int\limits_{\tau}^{\infty}h(y)dy\right)u(\tau)d\tau\right)^q w(t) dt\Bigg)^{\frac{1}{q}}\leq C\Big(\int\limits_{0}^{\infty}h^p(s)v(s)ds\Big)^{\frac{1}{p}} \label{1}
\end{equation}
for all $h\in \mathfrak{M}^+(0,\infty)$.

Using the Fubini theorem  for non-negative functions, we have
\begin{eqnarray*}
\int\limits_{0}^{s}u(\tau)\int\limits_{\tau}^{\infty}h(t)dt dy&=& \int\limits_{0}^{s}U(\tau)h(\tau)+U(s)\int\limits_{s}^{\infty}h(\tau)d\tau. \label{2}
\end{eqnarray*}

Therefore the inequality (\ref{1}) is equivalent with following inequality

\begin{equation}\left(\int\limits_{0}^{\infty}\left({\sup_{t<s<\infty}}\frac{1}{\varphi(s)}\left(\int\limits_{0}^{s} U(\tau) h(\tau)d\tau+U(s)\int\limits_{s}^{\infty}h(\tau)d\tau \right)\right)^qw(t)dt\right)^{\frac{1}{q}}\leq C\left(\int\limits_{0}^{\infty}h^p(\tau)v(\tau)d\tau\right)^{\frac{1}{p}}.
\label{3}  
\end{equation}

Throughout the paper, we always denote by $c$ or $C$ a positive constant which is
independent of the main parameters, but it may vary from line to line. However a constant
with subscript such as $С_1$ does not change in different occurrences.

\begin{theorem} \label{t main}
Let  $q\in(0,\infty)$, $p\in(1,\infty)$, and let $u, v, w$ be weights on $(0,\infty)$. $\varphi$ is $U$ -quasiconcave function on $(0, \infty)$, Then there exists a constant $C>0$ such that the inequality (\ref{3}) holds for all $f\in \mathfrak{M}^+(0,\infty)$  if and only if one of the following conditions is satisfied:

\rm(i) $1 < p \le q $,
\begin{equation*}
C_1 :=  \sup_{ t \in (0, \infty)} \left(\int_0^\infty \frac{w(s)ds}{(\varphi(s)+\varphi(t))^q}\right)^{\frac{1}{q}} \left(\int_0^\infty \frac{ U^{p'}(t) U^{p'}(s)}{ U^{p'}(s) +U^{p'}(t)} v^{1-p'}(s)ds \right)^{\frac{1}{p'}}<\infty,
\end{equation*}

\rm(ii)  $1<p$, $q<p$,
\begin{equation*}
C_2 :=  \left(\int_0^\infty \left(\int_0^\infty \frac{w(s)ds}{(\varphi(s)+\varphi(t))^q}\right)^{\frac{r}{q}}d\nu_p(t)\right)^{\frac{1}{r}}<\infty,
\end{equation*}
where $\nu_p$ is the representation measure of
\[
\varphi^r(t)\sup_{s\in(t,\infty)}\frac{1}{\varphi^r(s)}\left(\int_0^\infty \frac{ U^{p'}(\tau) U^{p'}(s)}{ U^{p'}(s) +U^{p'}(\tau)} v^{1-p'}(\tau)d\tau \right)^{\frac{r}{p'}},\]
i.e.
\[ \varphi^r(t)\sup_{s\in(t,\infty)}\frac{1}{\varphi^r(s)}\left(\int_0^\infty \frac{ U^{p'}(\tau) U^{p'}(s)}{ U^{p'}(s) +U^{p'}(\tau)} v^{1-p'}(\tau)d\tau \right)^{\frac{r}{p'}} =
\int_0^\infty \frac{\varphi^r(t)}{\varphi^{r}(s) +\varphi^{r}(t)} \nu_p(s)ds.
\]

Moreover, the best constant in the inequality (\ref{3}) satisfies
\begin{displaymath}
C\approx \left\{ \begin{array}{ll}
C_1 & \textrm{in case  (i),}\\
C_2 & \textrm{in case  (ii).}\\
\end{array} \right.
\end{displaymath}
\end{theorem}

The proof of Theorem \eqref{t main} is essentially based on the following theorem.

\begin{theorem}\label{disc.char. 3}
Let  $q\in(0,\infty)$, $p\in(1,\infty)$, and let $u, v, w$ be weights on $(0,\infty)$. $\varphi$ is $U$ -quasiconcave function on $(0, \infty)$, Then there exists a constant $C>0$ such that the inequality (\ref{3}) holds for all $f\in \mathfrak{M}^+(0,\infty)$ if and only if one of the following conditions is satisfied:

\rm(i) $1<p \leq q$,
\begin{align*}
A_1& := \sup_{k\in Z} \sup_{  x_k\le t \le  x_{k+1}} \frac{G(t)}{\varphi(t)} \left(\int_{x_k}^t U^{p'}(s) v^{1-p'}(s)ds +U^{p'}(t)\int_t^{x_{k+1}} v^{1-p'}(s)ds, \right)^{\frac{1}{p'}}<\infty,
\end{align*}

\rm(ii)  $1 < p $,  $q<p$,

\begin{equation*}
A_2 := \left(\sum_{k\in Z} \left(\sup_{  x_k\le  t \le  x_{k+1}} \frac{G(t)}{\varphi(t)} \left(\int_{x_k}^t U^{p'}(s) v^{1-p'}(s)ds +U^{p'}(t)\int_t^{x_{k+1}} v^{1-p'}(s)ds \right)^{\frac{1}{p'}}\right)^r\right)^{\frac{1}{r}}<\infty,
\end{equation*}
where $\{x_k\}_{k=N}^{M+1}$ is discretization sequence for $G$ and
\begin{equation*} \label{34}
G(t)=\int_{0}^{t}w(s)ds+\varphi(t)\int_{t}^{\infty}\varphi^{-1}(s)w(s)ds.
\end{equation*}

Moreover, the best constant in the inequality (\ref{3}) satisfies
\begin{displaymath}
C\approx \left\{ \begin{array}{ll}
A_1 & \textrm{in case  (i),}\\
A_2 & \textrm{in case  (ii).}
\end{array} \right.
\end{displaymath}
\end{theorem}

We present the following definitions and auxiliary statements from \cite{1, 2, 3, 4, 5} that are used to prove the indicated theorems.

\begin{definition}\label{def13} [2]
Let $\varphi$ be a continuous strictly increasing function on
$[0,\infty)$ such that $\varphi(0)=0$ and $\lim_{t\rightarrow\infty}\varphi(t)=\infty$. Then we say that $\varphi$ is admissible.
\end{definition}

Let $\varphi$ be an admissible function. We say that a function $h$ is $\varphi$-quasiconcave if $h$ is equivalent to an increasing function on $[0,\infty)$ and $\frac{h}{\varphi}$ is equivalent to a decreasing function on $(0,\infty)$. We say that a $\varphi$-quasiconcave function $h$ is non-degenerate if
\begin{equation}\lim_{t\rightarrow 0+}h(t)
=\lim_{t\rightarrow\infty}\frac{1}{h(t)}= \lim_{t\rightarrow\infty}\frac{h(t)}{\varphi(t)} =\lim_{t\rightarrow 0+}\frac{\varphi(t)}{h(t)}=0.
\end{equation}

The family of non-degenerate $\varphi$ -quasiconcave functions will be denoted by $\Omega_\varphi$.

\begin{definition}\label{def14} [3] Assume that $\varphi$ is admissible and $h \in \Omega_{\varphi}$. We say that $\left\{\mu_k\right\}_{k \in \mathbb{Z}}$ is a discretizing sequence for $h$ with respect to $\varphi$ if

(i) $\mu_0=1$ and $\varphi\left(\mu_k\right)\upuparrows$;

(ii) $h\left(\mu_k\right) \upuparrows$ and $\frac{h\left(\mu_k\right)}{\varphi\left(\mu_k\right)}\downdownarrows$;

(iii) there is a decomposition $\mathbb{Z}=\mathbb{Z}_1 \cup \mathbb{Z}_2$ such that $\mathbb{Z}_1 \cap \mathbb{Z}_2=\emptyset$ an for every $t \in\left[\mu_k, \mu_{k+1}\right]$,
$$
\begin{gathered}
h\left(\mu_k\right) \approx h(t) \quad \text { if } \quad k \in \mathbb{Z}_1, \\
\frac{h\left(\mu_k\right)}{\varphi\left(\mu_k\right)} \approx \frac{h(t)}{\varphi(t)} \quad \text { if } \quad k \in \mathbb{Z}_2 .
\end{gathered}
$$
\end{definition}

\begin{definition}\label{def15}
Let $\varphi$ be an admissible function and let $\nu$ be a non-negative
Borel measure on $[0,\infty)$. We say that the function $h$ defined by
$$
h(t)=\varphi(t)\int_{[0,\infty)}\frac{d\nu(s)}{\varphi(s)+\varphi(t)},~~t\in(0,\infty),
$$
is the fundamental function of the measure $\nu$ with respect to
$\varphi$. We will also say that $\nu$ is a representation measure of
$h$ with respect to $\varphi$.

We say that $\nu$ is non-degenerate if the following conditions are
satisfied for every $t\in(0,\infty)$:
$$\int_{[0,\infty)}\frac{d\nu(s)}{\varphi(s)+\varphi(t)}<\infty,~t\in (0,\infty) ~~~\mbox{and}~~~ \int_{[0,1]}\frac{d\nu(s)}{\varphi(s)}=\int_{[1,\infty)}d\nu(s)=\infty.$$
\end{definition}

\begin{lemma}\label{lem6} 
Assume that $\varphi$ is an admissible function, $f\in\Omega_\varphi$,
$\nu$ is a non-negative non-degenerate Borel measure on $[0,\infty)$ and
$h$ is the fundamental function of $\nu$ with respect to $\varphi$. If
$\{x_k\}$ is a discretizing sequence for $h$ with respect to
$\varphi$, then
$$ \int_{[0,\infty)}\frac{f(t)}{\varphi(t)}d\nu(t)\approx
\sum_{k\in Z}\left(\frac{f(x_k)}{\varphi(x_k)}\right)
\varphi(x_k). $$
\end{lemma}

\begin{lemma}\label{lem3}  \label{discr}
Assume that $\varphi$ is an admissible function, $\nu$ is a nondegenerate nonnegative Borel measure on $[0,\infty)$, $h$ is the fundamental function of $\nu$ and $f\in\mathfrak{M}^+(0,\infty)$. If $\{x_k\}$ is a discretizing sequence for $h$ with respect to $\varphi$, then
\begin{align*}
\int\limits_{[0,\infty)}\sup_{y\in(0,\infty)}\frac{|f(y)|}{\varphi(x)+\varphi(y)}d\nu(x)&
\approx\sum_{k\in\mathbb{Z}} \left(\sup_{y\in(0,\infty)} \frac {|f(y)|}{\varphi(x_k)+\varphi(y)}\right)\varphi(x_k)\\
&\approx\sum_{k\in\mathbb{Z}}\left(\varphi^{-1}(x_k)\sup_{x_{k-1}\leq y<x_k}|f(y)|+\sup_{x_{k}\leq y<x_{k+1}}|f(y)|\varphi^{-1}(y)\right)h(x_k)\\
&\approx\sum_{k\in\mathbb{Z}}\sup_{x_{k}\leq y<x_{k+1}}|f(y)|\varphi^{-1}(y)h(y).
\end{align*}
\end{lemma}

\textbf{Acknowledgement:}
 The research of A.N. Abek, A.Gogatishvili, was supported by the grant Ministry of Science and Higher Education of the Republic of Kazakhstan (project no: AP22686420).

\begin{thebibliography}{00}\label{ref:ref}

\bibitem{1} Evans W.D., Gogatishvili A., Opic B., “Weighted Inequalities Involving p-quasiconcave Operators”, World Scientific Publishing Co. Pte. Ltd., Hackensack, NJ, 2018.
\bibitem{2} Gogatishvili A., Mustafayev R.CH., Persson L.E., “Some New Iterated Hardy-Type Inequalities”, Journal of Function Spaces and Applications, (2012), 1-30.
\bibitem{3} Gogatishvili A., Pick L., “Discretization and anti-discretization of rearrangement-invariant norms”, Publ.Mat., Bare, (2003), 311-358.
\bibitem{4} Гогатишвили А., Степанов В.Д., “Редукционные теоремы для весовых интегральных неравенств на конусе монотонных функций”, УсМатНаук, 2013.
\bibitem{5} Bokayev N.A., Gogatishvili A., Abek A.N., “Cones of monotone functions generated by a generalized fractional maximal function”, TWMS,J.Pure Appl. Math., 15:1 (2024), 127-141.
\end{thebibliography}
\end{document}



